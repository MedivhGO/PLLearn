\documentclass{article}
\usepackage{ctex}
%\title{用户自定义打桩}
%\author{李敬奇}
%\date{\today}
\begin{document}
	%\maketitle
	\begin{Huge} \centerline{ \textbf{用户自定义打桩} } \end{Huge}
	
	\begin{large}
		\centerline{\today}
	\end{large}
	%\tableofcontents
	%\newpage
	\section{概述}
		\subsection{版本标识}
			本设计是第一版
			%\begin{itemize}
			%	\item hello
			%	\item world
			%\end{itemize}
			
		%	\begin{enumerate}
			%	\item abc
			%	\item bcd
			%\end{enumerate}
		\subsection{系统目标}
			对打桩模块进行功能上的扩展,使得用户能够自定义桩函数功能,提高覆盖率。
	\section{总体设计}
		\subsection{需求规定}
			

			
		\subsection{输入输出设计}
			\subsubsection{输入输出流图}
			
			\subsubsection{输入输出说明}
			
		\subsection{总体架构}
			\subsubsection{系统结构}
			位于unittest.util包下,用于配置打桩的方式 \\
			\begin{itemize}
			 \item stubGenerateMode=0表示随机打桩;
			 \item stubGenerateMode=1表示基于函数摘要打桩;
			\item stubGenerateMode=2表示基于函数摘要和路径打桩;
			\item stubGenerateMode=3表示用户自定义打桩函数;
			\end{itemize}
			
				
			\subsubsection{主要类}
				
			
				\begin{itemize}
				 	\item Config.java
					\item StubGenerate.java
					\item     
					unittest.teststub.generate.symboltableExpand 符号表扩展模块
					
					\item SCOPEFIRSTVISITOR.java
					
					\item unittest.teststub.generate.mustStubedFinding 需要打桩变量和函数查找模块
				
				\end{itemize}
	
			\subsubsection{子模块说明}		
		\subsection{处理流程}
			
	\section{接口设计}
		\subsection{外部接口}
			
			\subsubsection{函数摘要接口}
				
			\subsubsection{可行路径选择模块接口}
			
			\subsubsection{打桩与动态运行模块接口}
			
			\subsubsection{打桩与插装模块接口}
			
		\subsection{内部接口}
				
	
			
\end{document}