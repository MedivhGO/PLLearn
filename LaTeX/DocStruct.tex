%导言区
\documentclass{book}

\usepackage{ctex}

%正文区
\begin{document}
	\tableofcontents %制作目录
	\chapter{绪论}
	\section{引言}
	英特尔32位架构(英语:Intel Architecture, 32-bit,缩写为IA-32),常被称为i386、x86-32或是x86,由英特尔公司推出的复杂指令集(CISC)架构,至今英特尔最受欢迎的处理器仍然采用此架构。它是x86架构的32位延伸版本,1985年首次应用在Intel 80386芯片中用来取代之前的x86 16位架构(x86-16),包括8086、80186与80286芯片。IA-32 中word(字)表示16位数据类型,32位称为双字数据类型,64位称为四字。
	%分段一般通过插入空行来实现
	它是x86架构的32位延伸版本,1985年首次应用在Intel 80386芯片中用来取代之前的x86 16位架构(x86-16),包括8086、80186与80286芯片。IA-32 中word(字)表示16位数据类型,32位称为双字数据类型,64位称为四字。
	
	 它是x86架构的32位延伸版本,1985年首次应用在Intel 80386芯片中用来取代之前的x86 16位架构(x86-16),包括8086、80186与80286芯片。IA-32 中word(字)表示16位数据类型,32位称为双字数据类型,64位称为四字。

	
	
	\section{实验方法}
	\chapter{实验与结果分析}
	\section{实验结果}
		\subsection{数据} %在book中不起作用
		\subsection{图表}
		\subsection{实验条件}
		\subsection{实验过程}
		\subsection{结果分析}
	\section{结论}	
	\section{致谢}

\end{document}