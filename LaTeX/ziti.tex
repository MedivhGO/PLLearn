\documentclass[10pt]{article}
\usepackage{ctex}

\newcommand{\myfont}{\textit{\textbf{\textsf{Fancy Text}}}}c %自己定义的字体
\begin{document}
	%字体设置(罗马字体,无衬线字体,打字机字体)
	%字体族设置
	\textrm{Roman Family}  \textsf{Sans Serif Family}  \texttt{Typewriter Family}
	
	 %可以使用大括号来限定字体的作用范围
	 
	 
	\rmfamily Roman Family  {\sffamily Sans Serif Family}   {\ttfamily Typewriter Family}
	
	{\sffamily who are you? you find self on everyone around. take you as the same as others!}
	%字体声明命令
	
	{\ttfamily Are you wiser than others? definitely no. in some ways,
	may it is true. What can you achieve? a luxurious house? a brillilant car? an admirable career? 
	who konws?}
	
	%字体系列设置{粗细,宽度}
	\textmd{Medium Series}  \textbf{Boldface Series}
	
	{\mdseries Medium Series}    {\bfseries Boldface Series}
	
	%字体形状设置{直立,斜体,伪斜体,小型大写}
	\textup{Upright Shape} \textit{Italic Shape}
	\textsl{Slanted Shape}  \textsc{Small Caps Shape}
	
	{\upshape Upright Shape}  {\itshape Italic Shape} 
        {\slshape Slanted Shape} {\scshape Small Caps Shape}
        
        %中文字体设置
        {\songti 宋体}  \quad {\heiti 黑体}    \quad {\fangsong 仿宋}
        \quad {\kaishu 楷书}
        
       中文字体的\textbf{粗体}黑体表示 与\textit{斜体}楷书表示
       
       %字体大小 是相对于标准字体来设置的,标准字体的设置在documentclass 中进行了设置一般只有[10,11,12pt]
       {\tiny Hello}\\
       {\scriptsize Hello}\\
       {\footnotesize Hello}\\
       {\small Hello}\\
       {\normalsize Hello}\\
       {\large Hello}\\
       {\Large Hello}\\
       {\LARGE Hello}\\
       {\huge Hello}\\
       {\Huge Hello}\\
       
       %中文字号设置命令
       \zihao{5} 你好! %  -0 一般是小初号
       
       
       \myfont
      
        
	

\end{document}